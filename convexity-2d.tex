\documentclass{beamer}

% It seems that \newif has problems with items in the beamer package
% To have proofs:
%\newcommand{\hiddenproof}[2]{#1}
% To remove proofs
\newcommand{\hiddenproof}[2]{#2}


\newcommand{\out}[1]{}





\usetheme{Lecture}

\title[RC -- Konveksnost - 2d]{Konveksna ovojnica v ravnini}
\subtitle{Lekcija 4}
\author[Sergio Cabello]{Sergio Cabello \\ \texttt{sergio.cabello@fmf.uni-lj.si}\\ FMF \\ Univerza v Ljubljani}
\date{}
\begin{document}



\begin{frame}
    \titlepage
\end{frame}


\begin{frame}
	\frametitle{Zavijanje darila}
	
	\begin{algorithm}{Gift-wrapping}{\qinput{Mno"zica to"ck $P$}\qoutput{Seznam ogli"s"c za $CH(P)$}}
		Najdemo najni"zjo to"cko $q_0$ od $P$;\\
		$i \qlet 0$;\\
		\qrepeat\\
				$i\qlet i+1$;\\
				$q_{i}$ je taka to"cka, da je $q_{i-1},q_{i}, q$ zasuk na desno za vsak $q\in P$;
		\quntil{$q_i=q_0$}\\
		\qreturn $q_0,q_1,\dots, q_{i-1}$;
	\end{algorithm}
\end{frame}

\begin{frame}
	\frametitle{Zavijanje darila -- Splo"sni polo"zaj}
	Kje imamo lahko probleme? \\ 
	Veliko to"ck je lahko kandidatinj ali pa so na premici.\\
	\bigskip
	\begin{algorithm}{Gift-wrapping}{}
		Najdemo \alert{najni"zjo to"cko} $q_0$ od $P$;\\
		$i \qlet 0$;\\
		\qrepeat\\
				$i\qlet i+1$;\\
				$q_{i}$ je taka to"cka, da je \alert{$q_{i-1},q_{i}, q$} zasuk na desno za vsak $q\in P$;
		\quntil{$q_i=q_0$}\\
		\qreturn $q_0,q_1,\dots, q_{i-1}$;
	\end{algorithm}
\end{frame}


\begin{frame}
	\frametitle{Gramahov algoritem - Prva opcija}
	Izgradimo $M$ in ga spreminjamo. Potrebujemo kazalce.
	\medskip
	\begin{algorithm}{Grahamov algoritem}{}
		Najdemo najni"zjo to"cko $p_0=p_*$;\\
		$(p_1,\dotsm p_{n-1})\qlet$ uredimo to"cke $P\setminus \{p_*\}$  po kotih od $p_*$;\\
		Naredimo kro"zni seznam $L=(p_0,p_1,\dots, p_{n-1})$\\
		$p\qlet p_0$; $q\qlet p_1$; $r\qlet p_2$;\\
		\qwhile{$q\neq p_0$}\\
			\qif{$p,q,r$ zasuk na desno}\\
				$p\qlet q$; $q\qlet r$; $r\qlet r.next$;\\
			\qelse bri"semo element $q$ od $L$;\\
				$q\qlet p$; $p\qlet p.prev$;
			\qfi
		\qelihw\\
		\qreturn seznam $L$ minus zadnja $p_0$;
	\end{algorithm}
	\smallskip
	Ali je vrstica 5 problemati"cna, "ce je $p_0,p_1,p_2$ zasuk na levo? 
\end{frame}


\begin{frame}
	\frametitle{Gramahov algoritem - Druga opcija}
	Ne izgradimo $M$ na za"cetku, temve"c uporabimo za delo sklad $M$. Naredimo ga lahko s tabelo.\\
	\medskip
	\begin{algorithm}{Grahamov algoritem}{}
		Poi"s"cemo najni"zjo to"cko $p_*$;\\
		$(p_1,\dotsm p_{n-1})\qlet$ urejamo to"cke $P\setminus \{p_*\}$  po kotih od $p_*$;\\
		$S$ prazen sklad; damo $p_*$ in $p_1$ na sklad;\\
		$i\qlet 2$,
		\qwhile{$i<n$}\\
			$q$\qlet vrh $S$; $p$\qlet element pod vrhom $S$;\\
			\qif{$p,q,p_i$ zasuk na desno} \\
				damo $p_i$ na sklad $S$; \\
				$i\qlet i+1$;\\
			\qelse odstranimo element $q$ s sklada $S$;
			\qfi
		\qelihw\\
		\qreturn elementi sklada $S$
	\end{algorithm}
\end{frame}


\end{document}
