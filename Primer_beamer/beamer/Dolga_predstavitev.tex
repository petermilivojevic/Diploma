\documentclass[12pt, hyperref={unicode}]{beamer}

\usepackage[T1]{fontenc}
\usepackage[utf8]{inputenc}
\usepackage[slovene]{babel}

\usepackage{pgfpages}
\usepackage{bookmark}
\usepackage{graphicx}%za vstavljanje slik
\usepackage{array}%za tabele
\usepackage{enumerate}
\usepackage{lmodern}
\usepackage{amsfonts}
\usepackage{amsmath}


\mode<presentation>

%tema
\usetheme{Berlin}
\usecolortheme{default}
\useinnertheme[shadows]{rounded}
\useoutertheme{infolines}
\setbeamertemplate{navigation symbols}{}
\beamertemplatenavigationsymbolsempty

%pisava
\usepackage{palatino}
\usefonttheme{serif}
%serif doda podaljšane črtice pri I

\newtheorem{definicija}{Definicija}
\newtheorem{izrek}{Izrek}
\newtheorem{trditev}{Trditev}
\newtheorem{posledica}{Posledica}
\newtheorem{lema}{Lema}
\newtheorem{dokaz}{Dokaz}
\newtheorem{domneva}{Domneva}


\title{(Osnovne) Igre ustvarjanja omrežja}
\subtitle{Kratka predstavitev diplome}
\author{Peter Milivojević}
\institute[FMF]{Fakulteta za matematiko in fiziko}
\date{11. \ januar \ 2024}

\begin{document}

% ===================================================================
\begin{frame}
    \titlepage
\end{frame}
% -------------------------------------------------------------------


% -------------------------------------------------------------------
\begin{frame}
   
  \frametitle{(Osnovne) Igre ustvarjanja omrežja}
  Omrežje lahko ustvarimo na več načinov in pri različnih optimizacijskih pogojih.
  Omrežje lahko ustvari centralna avtoriteta in tako doseže socialni optimum ali pa
  več sebičnih igralcev, kjer vsak poskuša doseči svoj sebični optimum v dani situaciji.
  V diplomski nalogi se bom pretežno ukvarjal z dvema verzijama osnovnih iger ustvarjanja omrežja:
  \vspace{1mm}
  \begin{itemize}
    \item Maksimalna oddaljenos
    \item Vsota oddaljenosti oz. povprečna oddaljenost
  \end{itemize}

\end{frame}
% -------------------------------------------------------------------

% -------------------------------------------------------------------
\begin{frame}
   
    \frametitle{Uporabne definicije}
    \begin{definicija}
    $Graf$ je urejen par G = (V, E), kjer je V neprazna množica točk grafa G in E množica povezav grafa G, pričemer je vsaka povezava par točk.
    \end{definicija}

    \begin{definicija}
    $Graf$ G je povezan, če za vsak par voclišč $u, v \in V(G)$ obstaja pot od $u$ do $v$.
    \end{definicija}

\end{frame}
% -------------------------------------------------------------------

% -------------------------------------------------------------------
\begin{frame}
   
    \frametitle{Uporabne definicije}
    \begin{definicija}
    Naj bo $G$ povezan graf in $v \in V(G)$. Stopnja točke $v$ je enaka vsoti števila povezav, ki imajo to točko za krajišče (in dvojnega števila zank v tej točki).
    Označimo jo z $deg(v)$.
    \end{definicija}
    
    \begin{definicija}
    Naj bo $G$ povezan graf in $u, v \in V(G)$. Razdalja $d(u, v)$ je dolžina najkrajše poti med vozliščema $u$ in $v$ (t.j. razdalja med $u$ in $v$) v grafu $G$.
    \end{definicija}

\end{frame}
% -------------------------------------------------------------------

% -------------------------------------------------------------------
\begin{frame}
   
    \frametitle{Uporabne definicije}
    \begin{definicija}
    Naj bo $G$ povezan graf. Premer grafa $G$ je definiran kot $\text{diam}(G) = \max_{u, v \in V(G)} d(u, v)$, kjer je $d(u, v)$ razdalja med vozliščema $u$ in $v$ v grafu $G$.
    \end{definicija}
    
    \begin{definicija}
    Naj bo $G$ povezan graf. Lokalni premer točke $v$ grafa $G$ je definiran kot $\text{diam}(G) = \max_{u \in V(G)} d(u, v)$, kjer je $d(u, v)$ razdalja med vozliščema $u$ in $v$ v grafu $G$.
    \end{definicija}
    
    \begin{definicija}
    Povezan graf $G$ ima prerezno vozlišče $v$, če graf $G - v$ ni povezan.
    \end{definicija}

\end{frame}
% -------------------------------------------------------------------

% -------------------------------------------------------------------
\begin{frame}
   
    \frametitle{Uporabne definicije}
    \begin{definicija}
    Povezanost po povezavah $\lambda(G)$ povezanega grafa $G$ je najmanjše število povezav,
    z odstranitvijo katerih postane graf $G$ nepovezan. Če je $\lambda(G) \geq k$,
    je graf $G$ po povezavah $k$-povezan.
    \end{definicija}

    \begin{definicija}
    Naj bo $G$ povezan graf z n vozlišči. Weinerjev indeks $W = W(G)$ je definiran
    kot vsota vseh razdalj med vozlišči.
    $$W(G) = \sum_{i=1}^{n} \sum_{j=1}^{i} d_{ij} = \frac{1}{2} \sum_{i=1}^{n} \sum_{j=1}^{n} d_{ij}$$
    kjer $d_{ij}$ označuje dolžino najkrajše poti med vozliščem $i$ in $j$.
    \end{definicija}

\end{frame}
% -------------------------------------------------------------------

% -------------------------------------------------------------------
\begin{frame}
   
    \frametitle{Definiciji ravnovesja}
    \begin{definicija}
    Graf je v \textit{ravnotežju glede na vsoto razdalj}, če za vsako povezavo $vw$ in
    za vsako vozlišče $w'$ zamenjava povezave $vw$ z povezavo $vw'$ ne zmanjša
    celotne vsote razdalj od vozlišča $v$ do vseh ostalih vozlišč.
    \end{definicija}
        
    \begin{definicija}
    Graf je v \textit{ravnotežju glede na maksimalno razdaljo}, če za vsako povezavo $vw$
    in za vsako vozlišče $w'$ zamenjava povezave $vw$ z povezavo $vw'$ ne zmanjša
    lokalnega premera vozlišča $v$. Nadalje odstranitev povezave $vw$ poveča
    lokalni premer vozlišča $v$.
    \end{definicija}

\end{frame}
% -------------------------------------------------------------------

% -------------------------------------------------------------------
\begin{frame}
   
    \frametitle{Kritičnost in stabilnost}  
    \begin{definicija}
    Naj bo $G$ povezan graf. Graf $G$ je \textit{kritičen za odstranitev povezave},
    če odstranitev katere koli povezave $uv \in E(G)$ poveča lokalni premer vozlišča $v$ in vozlišča $u$.
    \end{definicija}
    
    \begin{definicija}
    Naj bo $G$ povezan graf. Graf $G$ je \textit{stabilen za dodajanje povezave},
    če dodajanje katere koli povezave $uv \in E(G)$ ne zmanjša lokalnega premera vozlišča $v$ in vozlišča $u$.
    \end{definicija}

\end{frame}
% -------------------------------------------------------------------

% -------------------------------------------------------------------
\begin{frame}

  Upoštevajte igro \( G = (N, S, u) \), ki jo določa množica igralcev \( N \), strategijski nabori \( S_i \) za vsakega igralca in uporabnosti \( u_i : S \to \mathbb{R}\), kjer je \( S = S_1 \times \ldots \times S_N\). To imenujemo tudi množica izidov. Lahko definiramo merilo učinkovitosti vsakega izida, ki mu pravimo funkcija blaginje Welf(s):\( s\to \mathbb{R}\). Naravni kandidati vključujejo vsoto uporabnosti igralcev (utilitarni cilj)\( Welf(s)=\sum_{i\in N}u_i(s)\), minimalno uporabnost (cilj pravičnosti ali egalitarni cilj)\( Welf(s)=\min_{i\in N}u_i(s)\).


Podmnožico Equil\( C_s\) lahko definiramo kot množico strategij v ravnovesju. Cena anarhije je potem definirana kot razmerje med optimalno 'centralizirano' rešitvijo in najslabšim 'ravnovesjem':

\end{frame}
% -------------------------------------------------------------------

% -------------------------------------------------------------------
\begin{frame}

  \frametitle{Ceni anarhije in stabilnosti}
  V delu se bomo ukvarjalise tudi z ceno anarhije (PoA) in ceno stabilnosti (PoS).
  Ceni anarhije in stabilnosti merita, kako se učinkovitost sistema poslabša zaradi sebičnega vedenja njegovih agentov.
  Cena anarhije je razmerje med vrednostjo/ceno, iz socialnega vidika (skupnega vidika vseh igralcev), najslabšega ravnovesja in socialno ceno optimalnega izida
  Cena stabilnosti igre pa je razmerje med najboljšo socialno ceno ravnovesja in socialno ceno optimalnega izida.
  V našem primeru sta definirani kot:
  \begin{definicija}
    $PoA = \frac{\text{Cena najslabšega ravnovesja}}{\text{Cena optimalne postavitve}} = \frac{\max_{s\in Equil} Cost(s)}{\min_{s\in S} Cost(s)}$
    $=\frac{\max_{s\in Equil} Cost(s)}{\min_{s\in S} Cost(s)}$
  \end{definicija}
  \begin{definicija}
    $PoA = \frac{\text{Cena najboljšega ravnovesja}}{\text{Cena optimalne postavitve}} = \frac{\min_{s\in Equil} Cost(s)}{\min_{s\in S} Cost(s)}$
  \end{definicija}

\end{frame}
% -------------------------------------------------------------------

% -------------------------------------------------------------------
\begin{frame}

  \frametitle{Drevesa: skupna vsota razdalj}
  \begin{izrek}
    Če je ravnovesni graf za skupno vsoto razdalj v preprosti igri ustvarjanja
    omrežja drevo, potem ima premer največ $2$ in je kot tak zvezda.
  \end{izrek}

  \includegraphics[width=0.3\textwidth]{drevo_1.png}
  \includegraphics[width=0.3\textwidth]{drevo_2.png}
  \includegraphics[width=0.3\textwidth]{drevo_3.png}
\end{frame}
% -------------------------------------------------------------------

% -------------------------------------------------------------------
\begin{frame}
   
  \frametitle{Drevesa: maksimalna razdalja}
  \begin{lema}
    V vsakem ravnovesnem grafu igre maksimalne razdalje se lokalni premer za
    katera koli $2$ poljubna vozlišča razlikuje največ za $1$.
  \end{lema}

  \begin{lema}
    Če ima ravnovesni graf za maksimalno razdaljo prerezno vozlišče, potem ima lahko
    samo ena izmed povezanih komponent $G - v$ vozlišče z razdaljo več kot $1$ od $v$.
  \end{lema}

  \begin{izrek}
    Če je ravnovesni graf za maksimalno razdaljo drevo, potem ima premer največ $3$.
  \end{izrek}

\end{frame}
% -------------------------------------------------------------------

% -------------------------------------------------------------------
\begin{frame}
  
  \frametitle{Skupna vsota razdalj: spodnje meje}
  \begin{lema}
    Za vozlišče $v$ z lokalnim premerom $2$, zamenjava sosednje povezave ne izboljša
    vsote razdalj od $v$ do vseh ostalih voclišč. Prav tako tudi svojega lokalnega premera ne more izboljšati.
  \end{lema}
  \begin{posledica}
    Vsak graf z premerom $2$ ali manj je v \textit{ravnotežju glede na vsoto razdalj}.
    In hkrati socialni optimum, zato je vena stabilnosti 1.
  \end{posledica}
  
\end{frame}

% -------------------------------------------------------------------

% -------------------------------------------------------------------
\begin{frame}
  
  \frametitle{Skupna vsota razdalj: spodnje meje}
  \begin{izrek}
    Naj bo $G$ graf z $n$ vozlišči in $m$ povezavami, potem je $W(G) = n^2 - n - m$, če in samo če je premer grafa $2$ ali manj.
  \end{izrek}
  \begin{posledica}
    Za vsak graf $G$ z $n$ vozlišči, $m$ povezavami in premerom večjim od $2$ velja:
    $$W(G) \geq n^2 - n - m + 1.$$
    Enačaj velja kadar ima graf $G$ natanko $2$ vozlišča z lokalnim premerom $3$ in vsa ostala vozlišča z lokalnim premerom $2$.
  \end{posledica}
  
\end{frame}

% -------------------------------------------------------------------

% -------------------------------------------------------------------
\begin{frame}
  
  \frametitle{Skupna vsota razdalj: spodnje meje}
  \begin{columns}
    \column{0.5\textwidth}
    \begin{izrek}
      Obstaja ravnovesni graf za skupno vsoto razdalj s premerom 3.
    \end{izrek}
    \column{0.5\textwidth}
    \includegraphics[width=1\textwidth]{vsota_premer_3.png}
  \end{columns}
  
\end{frame}

% -------------------------------------------------------------------

% -------------------------------------------------------------------
\begin{frame}
   
  \frametitle{Skupna vsota razdalj: zgornja meja $2^{O(\sqrt{\lg n})}$}
  \begin{izrek}
    Vsak ravnovesni graf za skupno vsoto razdalj ima premer $2^{O(\sqrt{\lg n})}$.
  \end{izrek}

  \begin{lema}
    Vsak ravnovesni graf za skupno vsoto razdalj ima premer največ $2 \lg n$ ali
    za vsako vozlišče $v$ obstaja povezava $xy$ kjer je $d(u, x) \leq \lg n$ in
    zamenjava povezave $xy$ zmanjša vsoto razdalj od $x$ za največ $2n(1 + \lg n)$.
  \end{lema}

  \begin{lema}
    V vsakem ravnovesnem grafu za skupno vsoto razdalj dodajanje poljubne povezave
    $uv$ zmanjša vsoto razdalj od $u$ za največ $5n \log n$.
  \end{lema}

\end{frame}
% -------------------------------------------------------------------

% -------------------------------------------------------------------
\begin{frame}

  \frametitle{Maksimalna razdalja}
  \begin{columns}
    \column{0.5\textwidth}
    \begin{izrek}
        Obstaja ravnovesni graf za maksimalno razdaljo z premerom $\Theta(\sqrt{n})$.
    \end{izrek}
    \column{0.5\textwidth}
    \includegraphics[width=1\textwidth]{Plagiat.png}
  \end{columns}
  
  

\end{frame}
% -------------------------------------------------------------------

% -------------------------------------------------------------------
\begin{frame}

  \frametitle{Povezava z grafi z enotsko razdaljo}
  \begin{izrek}
    Vsak ravnovesni graf za skupno vsoto razdaljo $G$ z $n \geq 24$ vozlišč in
    premerom $d > 2 \lg n$ inducira podgraf z $\epsilon\text{-skoraj-enotno-razdaljo } G'$
    z $n$ vozlišči in premerom $\Theta\left(\frac{\varepsilon d}{\lg n}\right)$ in
    podgraf z $\epsilon\text{-enotno-razdaljo } G'$ z $n$ vozlišči in premerom
    $\Theta\left(\frac{\varepsilon d}{\lg^2 n}\right)$.
  \end{izrek}

  \begin{domneva}
    Razdaljno skoraj enotni grafi imajo premer $O(\lg n)$.
  \end{domneva}


\end{frame}
% -------------------------------------------------------------------




\end{document}